\documentclass[10pt]{article}
\usepackage[margin=1in]{geometry}
\usepackage{framed}
\usepackage{amsmath}
\usepackage{graphicx}

\begin{document}

\begin{flushright}
	Name: \underline{\hspace{3cm}} \\
	MATH55 Section \underline{\hspace{0.5cm}} \\
	Homework 13 \\
	Due Tue. 4/2
\end{flushright}

% Graphs
% Homework, 3rd Edition — 47.7, 47.12, 47.17, 47.21(d,e).

\begin{framed}
	\textbf{47.7} Imagine creating a map on your computer screen. 
	This map wraps around the screen in the following way. A line that moves off the right side 
	of the screen instantly reappears at the corresponding position on the left. 
	Similarly, a line that drops off the bottom of the screen instantly reappears at the 
	corresponding position at the top. Thus it is possible to have a country on this map 
	that has a little section on the left and another little section on the right of the screen, 
	but is still in one piece.\\

	\noindent Devise such a computer-screen map that requires more than four colors.\\
	Try to create such a map that requires seven colors! (It is possible.)
\end{framed}

\pagebreak

\begin{framed}
	\textbf{47.12} Recall the university examination-scheduling problem.
	Create a list of courses and students such that more than four final 
	examination periods are required.
\end{framed}

\pagebreak

\begin{framed}
	\textbf{47.17} Let $G$ be an $r$-regular graph with $n$ vertices and $m$ edges. Find (and prove) a simple
	algebraic relation between $r$, $n$, and $m$.
\end{framed}

\pagebreak

\begin{framed}
	\textbf{47.21 (d,e)} What does it mean for two graphs to be the same? Let $G$ and $H$ be graphs. 
	We say that $G$ is isomorphic to $H$ provided there is a bijection $f: V(G) \rightarrow V(H)$ 
	such that for all $a, b \in V(G)$ we have $a ~ b$ (in $G$) if and only if $f(a) ~ f(b)$ (in $H$). 
	The function $f$ is called an isomorphism of $G$ to $H$.
	We can think of $f$ as renaming the vertices of $G$ with the names of the vertices in $H$ in a way that 
	preserves adjacency. Less formally, isomorphic graphs have the same drawing (except for the names of 
	the vertices).\\

	\textbf{d.} Give an example of two graphs that have the same number of vertices and the same number of 
	edges but that are not isomorphic.\\

	\includegraphics{hw13-graph-figure.png}

	\textbf{e.} Let $G$ be the graph whose vertex set is $\{1, 2, 3, 4, 5, 6\}$. In this graph, there 
	is an edge from $v$ to $w$ if and only if $v-w$ is odd. Let $H$ be the graph in the figure. 
	Find an isomorphism $f: V(G) \rightarrow V(H)$.
\end{framed}

\end{document}