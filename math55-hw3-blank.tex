\documentclass[10pt]{article}
\usepackage[margin=1in]{geometry}
\usepackage{framed}

\newcommand{\cmp}[1]{\overline{#1}}

\begin{document}

\begin{flushright}
	Name: \underline{\hspace{3cm}} \\
	MATH55 Section \underline{\hspace{0.5cm}} \\
	Homework 3 \\
	Due Tuesday 2/5
\end{flushright}

\begin{framed}
    \textbf{12.22} Let $A$ be a set. The complement of $A$, denoted $A$, is the set of all objects that are not in $A$. STOP! 
    This definition needs some amending. Taken literally, the complement of the set $\{1, 2, 3\}$ 
    includes the number 􏰏5, the ordered pair (3, 4), and the sun, moon, and stars! After all, it says “. . . all objects that are not in A.” 
    This is not what is intended.

    When mathematicians speak of set complements, they usually have some overarching set in mind. 
    For example, during a given proof or discussion about the integers, if $A$ is a set containing just integers, 
    $A$ stands for the set containing all integers not in $A$.

    If $U$ (for “universe”) is the set of all objects under consideration and $A\subseteq􏰁 U$ , then the complement of $A$ is the set of 
    all objects in $U$ that are not in $A$. In other words, $\cmp{A} = U 􏰏- A$. Thus $\cmp{\emptyset} = U$.

    Prove the following about set complements. Here the letters $A$, $B$, and $C$ denote subsets of a universe set $U$.\\

    \indent \textbf{a.} $A=B$ if and only if $\cmp{A} = \cmp{B}$.\\
    \indent \textbf{b.} $\cmp{\cmp{A}} = A$.\\
    \indent \textbf{c.} $\cmp{A \cup B \cup C} = \cmp{A} \cap \cmp{B} \cap \cmp{C}$.\\

    The notation $\cmp{A}$ is handy, but it can be ambiguous. 
    Unless it is perfectly clear what the “universe” set $U$ should be, 
    it is better to use the set difference notation rather than complement notation.
\end{framed}

\pagebreak

\begin{framed}
    \textbf{13.1} Give an alternative proof of Proposition 13.1 in which you use list counting instead of subset counting.\\

    Proposition 13.1: Let $n$ be a positive integer. Then
    $$
    2^0+2^1+...+2^{-1} = 2^n-1
    $$
\end{framed}

\pagebreak

\begin{framed}
    \textbf{13.3} Substituting $x = 3$ into your expression in the previous problem yields
    $$
    2\times3^0+2\times3^1+2\times3^2+...+2\times3^{n-1} = 3^n-1
    $$
    Prove this equation combinatorially.
    Next, substitute $x = 10$ and illustrate the result using ordinary base-10 numbers.
\end{framed}

\pagebreak

\begin{framed}
    \textbf{13.5} Let $n$ be a positive integer. Give a combinatorial proof that $n^2 = n(n-1)+n$.
\end{framed}

\end{document}