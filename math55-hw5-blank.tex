\documentclass[10pt]{article}
\usepackage[margin=1in]{geometry}
\usepackage{framed}
\usepackage{amsmath}

% define /multichoose command which works differently for inline math mode and full math mode.
\def\multichooseA#1#2{\left(\kern-.3em\binom{#1}{#2}\kern-.3em\right)}
\def\multichooseB#1#2{\big(\kern-.3em\binom{#1}{#2}\kern-.3em\big)}
\newcommand{\multichoose}[2]{\mathchoice{\multichooseA{#1}{#2}}{\multichooseB{#1}{#2}}{\multichooseA{#1}{#2}}{\multichooseA{#1}{#2}}}

\begin{document}

\begin{flushright}
	Name: \underline{\hspace{3cm}} \\
	MATH55 Section \underline{\hspace{0.5cm}} \\
	Homework 5 \\
	Due Tue. 2/12
\end{flushright}

\begin{framed}
    \textbf{17.33} A poker hand consists of 5 cards chosen from a standard deck of 52 cards.
    There are a variety of special hands that one can be dealt in poker. For
    each of the following types of hands, count the number of hands that have that type.\\

    \indent \textbf{a.} Four of a kind: The hand contains four cards of the same numerical value (e.g., four
    jacks) and another card.\\
    \indent \textbf{b.} Three of a kind: The hand contains three cards of the same numerical value and two
    other cards with two other numerical values.\\
    \indent \textbf{c.} Flush: The hand contains five cards all of the same suit.\\
    \indent \textbf{d.} Full house: The hand contains three cards of one value and two cards of another value.\\
    \indent \textbf{e.} Straight: The five cards have consecutive numerical values, such as 7-8-9-10-jack.
    Treat ace as being higher than king but not less than 2. The suits are irrelevant.\\
    \indent \textbf{f.} Straight flush: The hand is both a straight and a flush.
\end{framed}

\pagebreak

\begin{framed}
    \textbf{18.13} Prove:
    $$
        \multichoose{n}{k} = \multichoose{k+1}{n-1}.
    $$
\end{framed}

\pagebreak

\begin{framed}
    \textbf{18.15} Let $n, k$ be positive integers, prove:
    $$
    \multichoose{n}{k} = \multichoose{n-1}{0} + \multichoose{n-1}{1}
    + \multichoose{n-1}{2} + ...  + \multichoose{n-1}{k}
    $$
\end{framed}

\pagebreak

\begin{framed}
    \textbf{18.16} Let $n,k$ be positive integers, prove:
    $$
    \multichoose{n}{k} = \multichoose{1}{k-1} + \multichoose{2}{k-1}
    +  ...  + \multichoose{n}{k-1}
    $$
\end{framed}

\end{document}