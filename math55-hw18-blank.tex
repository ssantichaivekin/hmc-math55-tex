\documentclass[10pt]{article}

\usepackage[margin=1in]{geometry}
\usepackage{framed}
\usepackage{graphicx}

\begin{document}

\begin{flushright}
    Name: \underline{\hspace{3cm}} \\
        MATH55 Section \underline{\hspace{0.5cm}} \\
        Homework 18 (Planar Graphs) \\
	Due Thurs. 4/18
\end{flushright}

\begin{framed}
    \textbf{53.7} For which values of $n$ is the $n$-cube $Q_n$ planar? Prove
    your answer.\\

    Definition of $Q_n$:\\
    Let $n$ be a positive integer. The \textit{n-cube} is a graph, denoted $Q_n$,
    whose vertices are the $2^n$ possible length-$n$ lists of 0s and 1s. For
    example, the vertices of $Q_3$ are 000, 001, 010, 011, 100, 101, 110, and
    111.\\
    Two vertices of $Q_n$ are adjacent if their lists differ in exactly one
    position. For example, in $Q_4$, vertices 1101 and 1100 are adjacent
    (they differ only in their fourth element), but 1100 and 0110 are not
    adjacent (they differ in positions 1 and 3).
\end{framed}

\pagebreak

\begin{framed}
    \textbf{53.8} The graph in the figure is known as \textit{Petersen's
    Graph}. Prove that it is nonplanar by finding either a subdivision of
    $K_5$ or a subdivision of $K_{3,3}$ as a subgraph.

    \includegraphics[scale=0.25]{image1hw18}
\end{framed}

\pagebreak

\begin{framed}
    \textbf{53.12} A \textit{Platonic} graph is a connected planar graph
    in which all vertices have the samee degree $r$ (with $3 \leq r \leq 5$)
    and in whose crossing-free embedding all faces have the same degree $s$
    (with $3 \leq s \leq 5$). Let $G$ be a platonic graph with $v$ vertices,
    $e$ edges, and $f$ faces.

    \textbf{a.} Prove that $vr=fs$. How is this quantity related to $e$?

    \textbf{b.} Prove that if $r=s=3$, then $v=f=4$. Conclude that $K_4$ is
    the only Platonic graph with $r=s=3$.

    \textbf{c.} Prove that
    \[e = \frac{2}{\frac{2}{r} + \frac{2}{s} - 1}\]

    \textbf{d.} In all, there are nine ordered pairs $(r, s)$ with
    $3 \leq r, s \leq 5$. Use the equation in part (c) to rule out
    the existence of Platonic graphs with some of these values.

    \textbf{e.} For the pairs $(r, s)$ that were not ruled out in part (d),
    find a Platonic graph with vertex degree $r$ and face degree $s$.
\end{framed}

\pagebreak

\begin{framed}
    \textbf{53.13} A soccer ball is formed by stitching together pieces of
    material that are regular pentagons and regular hexagons. The lengths
    of the sides of these polygons are all the same, so the edges match up
    exactly. Each corner of a polygon is the meeting place for exactly three
    polygons.

    Prove that there must be exactly 12 pentagons.
\end{framed}

\end{document}

