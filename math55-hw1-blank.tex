\documentclass[10pt]{article}
\usepackage[margin=1in]{geometry}
\usepackage{framed}

\begin{document}

\begin{flushright}
	Name: \underline{\hspace{3cm}} \\
	MATH55 Section \underline{\hspace{0.5cm}} \\
	Homework 1 \\
	Due Tuesday 1/29
\end{flushright}

\begin{framed}
	\textbf{8.9.} In how many ways can a black rook and a white rook be
	placed on different squares of a chess board such that neither is
	attacking the other? (In other words, they cannot be in the same row or
	the same column of the chess board. A standard chess board is $8 \times
		8$.)
\end{framed}

\pagebreak

\begin{framed}
	\textbf{8.16.} A padlock has the digits 0 through 9 arranged in a
	circle on its
	face. A combination for this padlock is four digits long. Because of
	the
	internal mechanics of the lock, no pair of consecutive numbers in the
	combination can be the same or one place apart on the face. For example
	0-2-7-1
	is a valid combination, but neither 0-4-4-7 (repeated digit 4) nor
	3-0-9-5
	(adjacent digits 0-9) are permitted. \\
	\indent How many combinations are possible?
\end{framed}

\pagebreak

\begin{framed}
	\textbf{10.11} Generalize the previous problem. Let $a$ and $b$ be
	integers and let $ A =
		\{x\:\epsilon\:Z : a | x\} $ and $ B = \{ x\:\epsilon\:Z: b |
		x\} $. \\
	\indent Find and prove necessary and sufficient condition for $ A
		\subseteq B $. In other words, given the notation developed,
	find and prove a theorem of the form ``$A \subseteq B$ if and only if
	\textit{some condition involving a and b}.''

\end{framed}

\pagebreak


\begin{framed}
	\textbf{10.13} Generalize the previous problem. Let $c$ and $d$ be
	integers and let $C = \{x\:\epsilon\:Z : x|c\}$ and $D =
		\{x\:\epsilon\:Z:x|d\}$. \\
	\indent Find and prove a necessary and sufficient condition for
	$ C \subseteq D $.
\end{framed}

\end{document}