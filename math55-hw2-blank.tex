\documentclass[10pt]{article}
\usepackage[margin=1in]{geometry}
\usepackage{framed}

\begin{document}

% Homework, 3rd Edition — 14.16, 15.16, 16.8, 16.13.

\begin{flushright}
	Name: \underline{\hspace{3cm}} \\
	MATH55 Section \underline{\hspace{0.5cm}} \\
	Homework 2 \\
	Due Thursday 1/31
\end{flushright}

\begin{framed}
	\textbf{14.16.} Let us say that two integers are \textit{near} one another provided the absolute value of their
    difference is 2 or smaller (i.e., the numbers are at most 2 apart). For example, 3 is near to 5, 10 is near to 9, but 8 is not near to 4.
    Let R stand for this is-near-to relation. Please do the following:\\

    \indent \textbf{a.} Write down $R$ as a set of ordered pairs. Your answer should look like this:\\
    $$ R = \{(x, y):...\} $$
    \indent \textbf{b.} Prove or disprove: $R$ is reflexive.\\
    \indent \textbf{c.} Prove or disprove: $R$ is irreflexive.\\
    \indent \textbf{d.} Prove or disprove: $R$ is symmetric.\\
    \indent \textbf{e.} Prove or disprove: $R$ is antisymmetric.\\
    \indent \textbf{f.} Prove or disprove: $R$ is transitive.
\end{framed}

\pagebreak

\begin{framed}
    \textbf{15.16} There is only one possible equivalence relation on a one-element set:
    If $A = \{1\}$, then $R = \{(1,1)\}$ is the only possible equivalence relation.
    There are exactly two possible equivalence relations on a two-element set:
    If $A=\{1, 2\}$, then $R_1 = \{(1,1),(2,2)\}$ amd $R_2 = \{(1,1),(1,2),(2,1),(2,2)\}$ are
    the only equivalence relations on A.\\

    \indent How many different equivalence relations are possible on a three-element set?
    ... on a four-element set?
\end{framed}

\pagebreak

\begin{framed}
	\textbf{16.8} Continued from the previous problem. Suppose six of these people are men, and the
    other six are women. In how many ways can they join hands for a circle dance, assuming
    they alternate in gender around the circle?\\

    (Last problem: Twelve people join hands for a circle dance. In how many ways can they do this?)
\end{framed}

\pagebreak

\begin{framed}
    \textbf{16.13} One hundred people are to be divided into ten discussion groups with ten people in each group.
    In how many ways can this be done?
\end{framed}

\end{document}