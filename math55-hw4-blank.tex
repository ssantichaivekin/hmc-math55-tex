\documentclass[10pt]{article}
\usepackage[margin=1in]{geometry}
\usepackage{framed}
\usepackage{amsmath}

\begin{document}

\begin{flushright}
	Name: Santi Santichaivekin \\
	MATH55 Section 3 \\
	Homework 4 \\
	Due Thu 2/7
\end{flushright}

\begin{framed}
    \textbf{17.16} Consider the following formula:
    $$
    k \binom{n}{k} = n\binom{n-1}{k-1}
    $$
    Give two different proofs. One proof should use the factorial formula for $\binom{n}{k}$ 
    (Theorem 17.12). The other proof should be combinatorial; develop a question that both sides of
    the equation answer.
\end{framed}

Factorial formula proof:
\begin{align*}
    k\binom{n}{k} &= k(\frac{n!}{k!(n-k)!})  && \text{: via factorial formula}\\
                  &= \frac{n(n-1)!}{(k-1)!(n-k)!}\\
                  &= \frac{n(n-1)!}{(k-1)!((n-1)-(k-1))!}\\
                  &= n\binom{n-1}{k-1} &&\text{: via factorial formula}\\ 
\end{align*}

\pagebreak

\begin{framed}
    \textbf{17.19} Let $n$ be natural number. Give a combinatorial proof of the following:
    $$
    \binom{2n+2}{n+1} = \binom{2n}{n+1} + 2\binom{2n}{n} + \binom{2n}{n-1}
    $$
\end{framed}

\pagebreak

\begin{framed}
    \textbf{17.21} Use Stirling's formula (see Exercise 9.7) to develop an approximation formula for 
    $\binom{2n}{n}$.\\
    Without using Stirling's formula, give a direct proof that $\binom{2n}{n} \leq 4^n$.\\

    Reference from Exercise 9.7: The Scottish mathematician James Stirling found an approximation formula for $n!$.
    Stirling’s formula is
    $$
        n! \approx \sqrt{2\pi n} n^n e^{-n}
    $$
    where $\pi = 3.13159...$ and $e = 2.71828...$. (Scientific calculators have a key that computes $e^x$;
    this key might be labeled ``exp x''.)
\end{framed}

\pagebreak

\begin{framed}
    \textbf{17.26} Prove: $\binom{n}{0}\binom{n}{n} + \binom{n}{1}\binom{n}{n-1} +
    \binom{n}{2}\binom{n}{n-2} + ... + \binom{n}{n-1}\binom{n}{1} + \binom{n}{n}\binom{n}{0} = \binom{2n}{n}$.
\end{framed}

\end{document}