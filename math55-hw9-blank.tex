\documentclass[10pt]{article}
\usepackage[margin=1in]{geometry}
\usepackage{framed}
\usepackage{amsmath}

\begin{document}

\begin{flushright}
	Name: \underline{\hspace{3cm}} \\
	MATH55 Section \underline{\hspace{0.5cm}} \\
	Homework 9 \\
	Due Tue. 2/26
\end{flushright}

% 24.16, 24.18, 25.5, 25.7.

\begin{framed}
	\textbf{24.16} Let $A$ and $B$ be finite sets and let $f : A \rightarrow B$. 
	Prove that any two of the following statements being true implies the third.\\
	\indent \textbf{a.} $f$ is one-to-one.\\
	\indent \textbf{b.} $f$ is onto.\\
	\indent \textbf{c.} $|A| = |B|$.
\end{framed}

\pagebreak

\begin{framed}
	\textbf{24.18} Suppose $f :A \rightarrow B$ is a bijection.
	Prove that $f^{-1} : B \rightarrow A$ is a bijection as well.
\end{framed}

\pagebreak

\begin{framed}
	\textbf{25.5} Let $(a_1,a_2,a_3,a_4,a_5)$ be a sequence of five distinct integers.
	We call such a sequence increasing if $a_1 <a_2 <a_3 <a_4 <a_5$ and decreasing if
	$a_1 >a_2 >a_3 >a_4 > a_5$. 
	Other sequences may have a different pattern of $<$s and $>$s. 
	For the sequence $(1,5,2,3,4)$ we have $1 < 5 > 2 < 3 < 4$. 
	Different sequences may have the same pattern of $<$s and $>$s between their elements.
	For example, $(1, 5, 2, 3, 4)$ and $(0, 6, 1, 3, 7)$ have the same pattern of $<$s and 
	$>$s as illustrated here:
	\begin{align*}
		1 < 5 > 2 < 3 < 4\\
		0 < 6 > 1 < 3 < 7
	\end{align*}
	Given a collection of 17 sequences of five distinct integers, prove that 2 of them have
	the same pattern of $<$s and $>$s.
\end{framed}

\pagebreak

\begin{framed}
	\textbf{25.7} Given a set of seven distinct positive integers, 
	prove that there is a pair whose sum or whose difference is a multiple of 10.

	You may use the fact that if the ones digit of an integer is 0, then that integer 
	is a multiple of 10.
\end{framed}

\end{document}