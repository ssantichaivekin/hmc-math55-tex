\documentclass[10pt]{article}
\usepackage[margin=1in]{geometry}
\usepackage{framed}
\usepackage{amssymb}

\begin{document}

\begin{flushright}
	Name: \underline{\hspace{3cm}} \\
	MATH55 Section \underline{\hspace{0.5cm}} \\
	Homework 8 \\
	Due Thu. 2/21
\end{flushright}

% 22.9, 22.14, 22.17, 22.18.

\begin{framed}
	\textbf{22.9} A group of people stand in line to purchase movie tickets. 
	The first person in line is a woman and the last person in line is a man. 
	Use proof by induction to show that somewhere in the line a woman is 
	directly in front of a man.
\end{framed}

\pagebreak

\begin{framed}
	\textbf{22.14} Let $A_1, A_2, ..., A_n$ be sets (where $n \geq 2$). 
	Suppose for any two sets $A_i$ and $A_j$ either $A_i \subseteq A_j$ or
	$A_j \subseteq A_i$.\\

	Prove by induction that one of these n sets is a subset of all of them.
\end{framed}

\pagebreak

\begin{framed}
	\textbf{22.17} A flagpole is n feet tall. 
	On this pole we display flags of the following types: 
	red flags that are 1 foot tall, blue flags that are 2 feet tall, 
	and green flags that are 2 feet tall. The sum of the heights of the flags 
	is exactly $n$ feet.\\

	Prove that there are $\frac{2}{3}2^n+\frac{1}{3}(-1)^n$ ways to display the flags.
\end{framed}

\pagebreak

\begin{framed}
	\textbf{22.18} Prove that every positive integer can be expressed as the sum 
	of distinct Fibonacci numbers.\\

	For example, $20 = 2 + 5 + 13$ where $2, 5, 13$ are, of course, Fibonacci numbers. 
	Although we can write $20 = 2 + 5 + 5 + 8$, this does not illustrate the result 
	because we have used 5 twice.
\end{framed}

\end{document}